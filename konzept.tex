%!TEX root = thesis.tex

\chapter{Konzept}
\label{chapter-konzept}

In der Philosophie des Existentialismus ist Langeweile ein Grundzustand der menschlichen Existenz.
In der neueren Philosophiegeschichte ist das Gefühl der Langeweile ebenso zum Thema geworden wie die Empfindungen des Ekels, der Angst oder der Verzweiflung.


\section{Quellen}

Wichtige Quellen zur Langeweile
\begin{compactitem}
  \item Das Standardwerk von Bellebaum \cite{StandardBelle},
  \item ein Beitrag in einem Sammlungsband \cite{ThGo},
  \item Expose vom Kruemmelmonster aus dem Berichtsband der Kekse JETZT Konferenz \cite{KrMol},
\end{compactitem}










\begin{figure}
  \centering
  \pgfimage[width=.5\textwidth]{Bild1}
  \caption[Langweiliges Bild]{Ein sehr sehr langweiliges Bild von Heinrich Ebel mit dem Titel Abend, entstanden 1895.}
  \label{fig-Bild1}
\end{figure}

\begin{figure}
  \centering
  \pgfimage[width=.5\textwidth]{Bild2}
  \caption[Langweiliges Screenshot]{Ein sehr sehr langweiliges Screenshot von einer bekannten Suchmaschine, gemacht vom forum Die Staemme}
  \label{fig-Bild2}
\end{figure}










