%!TEX root = thesis.tex

\chapter{Philosophie und Literatur}
\label{chapter-Soziologie}

Elisabeth Prammer definiert in ihrer soziologischen Analyse mehrerer Biographien, zur Untersuchung des Boreout-Syndroms, Langweile als ein Ergebnis von Zeitverknappung: unter den verschiedenen Möglichkeiten, die Zeit zu nutzen, kann die gewünschte nicht gewählt werden weil der betreffende Mensch aus einem bestimmten Grund daran gehindert ist, frei zu wählen.[3] Wenn dies begleitet wird, von der Abwesenheit von Neugier, Interesse und Kreativität, gleiche der Zustand einer Erschöpfungsdepression, die von schneller Erschöpfung und Gefühlen der Ohnmacht begleitet wird.[4] Existenzielle Langeweile ginge in den Lebenssinn über, während gewöhnliche Langeweile bedeute, im Augenblick nicht ausgelastet zu sein.[5] Der Gegensatz zur Langeweile sei der Flow-Zustand, bei dem ein Mensch sein Tun nicht in Bezug zur Zeit setzt, sondern ganz darin aufgeht

\section{Auftretende Probleme}

Zum Problem wird Langweile nach Prammer, weil sie abgewertet wird und der moderne Mensch unter dem Druck steht, seine Zeit sinnvoll nutzen zu müssen.[6] Aktivität an sich, verhindert Langweile nicht auf jeden Fall. Denn wer kein Interesse für seine Tätigkeit habe, langweile sich bei ihr, so Prammer.[7] Routine führt dann nicht zu Langweile, wenn sie Sicherheit vermittelt und zu Ausführung der Tätigkeit notwendig ist – solange sie nicht davon abhält, Neues zu entdecken und zu erleben.[8] Im Arbeitsleben ist Langeweile besonders problematisch, wenn sie mit dem Verlust des Gefühls für den Sinn der eigenen Tätigkeit einhergeht, weil diese vielleicht nur aufgrund einer extrinsischen Motivation (bzw. eines ökonomischen Zwangs) ausgeübt wird

