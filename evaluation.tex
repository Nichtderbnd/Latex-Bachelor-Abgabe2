%!TEX root = thesis.tex

\chapter{Philosophie und Literatur}
\label{chapter-evaluation}

In der Philosophie des Existentialismus ist Langeweile ein Grundzustand der menschlichen Existenz.
In der neueren Philosophiegeschichte ist das Gefühl der Langeweile ebenso zum Thema geworden wie die Empfindungen des Ekels, der Angst oder der Verzweiflung.
So analysierte unter anderem der Philosoph Martin Heidegger die Langeweile. In seiner Antrittsvorlesung 1929 behandelte er die Langeweile als ein Sich-Befinden des Seienden im Ganzen, das an sich nie absolut zu erfassen sei. Die tiefe Langeweile sei einem schweigenden Nebel vergleichbar, der alle Dinge in eine merkwürdige Gleichgültigkeit zusammenrücke. Die Langeweile offenbare das Seiende im Ganzen.[1] Außerdem teilte er die Langeweile in verschiedene Phasen ein. Dabei spezifizierte er (sinngemäß nach dem Buch Ein Meister aus Deutschland von Rüdiger Safranski) die drei Phasen wie folgt:
\begin{compactitem}[--]
  \item Von Etwas gelangweilt werden: Die Langeweile hat einen identifizierbaren Grund, dem die Langeweile zugeschrieben werden kann.
  \item Sich bei etwas langweilen: Die Langeweile kommt sowohl von innen als auch von außen und kann nicht mehr eindeutig einem Grund zugeordnet werden.
  \item Die gänzlich anonyme Langeweile: Sie besitzt keinen erkennbaren Grund und ist bezugslos.
\end{compactitem}

\section{Blaise Pascal Gedankengang}

Der Literat, Naturwissenschaftler und Philosoph Blaise Pascal schrieb zur Langeweile, der er keinen Nutzen entnehmen konnte, folgendes: \\
„Nichts ist so unerträglich für den Menschen, als sich in einer vollkommenen Ruhe zu befinden, ohne Leidenschaft, ohne Geschäfte, ohne Zerstreuung, ohne Beschäftigung. Er wird dann sein Nichts fühlen, seine Preisgegebenheit, seine Unzulänglichkeit, seine Abhängigkeit, seine Ohnmacht, seine Leere. Unaufhörlich wird aus dem Grund seiner Seele der Ennui aufsteigen, die Schwärze, die Traurigkeit, der Kummer, der Verzicht, die Verzweiflung."

